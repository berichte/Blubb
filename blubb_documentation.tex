\documentclass[12pt,a4paper,oneside]{report}
\usepackage[utf8]{inputenc}
\usepackage{graphicx, titlesec}
\usepackage{hyperref}
\usepackage{url}
%\usepackage{cite}
%\usepackage{natbib}
\usepackage[autostyle]{csquotes}

\usepackage{floatrow}

\usepackage[
	backend=biber,
	style=ieee,
	natbib=true]{biblatex}
\addbibresource{thesisBib.bib}

\usepackage{listings}
\usepackage{color}

\definecolor{dkgreen}{rgb}{0,0.6,0}
\definecolor{gray}{rgb}{0.5,0.5,0.5}
\definecolor{mauve}{rgb}{0.58,0,0.82}
\definecolor{lightgray}{rgb}{.9,.9,.9}
\definecolor{darkgray}{rgb}{.4,.4,.4}
\definecolor{purple}{rgb}{0.65, 0.12, 0.82}

\lstdefinelanguage{JavaScript}{
  keywords={typeof, new, true, false, catch, function, return, null, catch, switch, var, if, in, while, do, else, case, break},
  keywordstyle=\color{blue}\bfseries,
  ndkeywords={class, export, boolean, throw, implements, import, this},
  ndkeywordstyle=\color{darkgray}\bfseries,
  identifierstyle=\color{black},
  sensitive=false,
  comment=[l]{//},
  morecomment=[s]{/*}{*/},
  commentstyle=\color{blue}\ttfamily,
  stringstyle=\color{mauve}\ttfamily,
  morestring=[b]',
  morestring=[b]"
}

\lstset{frame=tb,
  language=Java,
  aboveskip=3mm,
  belowskip=3mm,
  showstringspaces=false,
  columns=flexible,
  basicstyle={\small\ttfamily},
  numbers=none,
  numberstyle=\tiny\color{gray},
  keywordstyle=\color{blue},
  commentstyle=\color{dkgreen},
  stringstyle=\color{mauve},
  breaklines=true,
  breakatwhitespace=true,
  tabsize=3,
  escapeinside=`` 
}


\titleformat{\chapter}
	{\normalfont\Large\bfseries}{\thechapter.}{1em}{}
\graphicspath{ {images/} }


\newcommand{\appname}{ablubb}
\newcommand{\blubb}{blubb}
\newcommand{\beapDB}{beapDB}
\newcommand{\beapServer}{beap server}

\newcommand{\code}[1]{\lstinline{#1}}

\begin{document}
\begin{lstlisting}
\end{lstlisting}
%\bibliographystyle{ieeetr}
%\citep[pp. 10-12]{}, \citet{}, \citep*{}
\title{
	{\huge \appname{} documentation}
	{\\ \large Hochschule Heilbronn}
}
\author{Benjamin Richter}
\date{\today}
\maketitle
\tableofcontents

\chapter{Introduction}

\section{Project description}

\section{Project management and communication}
%comment
project management \citep{mihaela2013measurement}. 


\section{General goals}


\chapter{Execution}

\section{Backlog}

\begin{itemize}
     \item As a User, I don't want a message I wrote but did not send to disappear.
     \item As a User, I want to shut down the application. 
     \item As a User, I want to choose my own password. 
     \item As a Admin, I want to open a new Admin thread.
     \item As a Manager, I want to open a new Management thread. 
     \item As a Manager, I want to show a TASK only to some Users. 
     \item As a User, I want to finish a TASK (mark it as finished). 
     \item As a User, I want to take a TASK.
     \item As a User, I want to view only TASKS from a thread or all threads.
     \item As a Manager, I want to post TASKS User can work on. 
     \item As a User, I want to view only QUESTIONS from a thread or from all threads. 
     \item As a User, I want to ask some other User a QUESTION and view the results as a chart. 
     \item As a User, I want to edit my own profile description. 
     \item As a User, I want to see profiles of others with description, status, profile picture and batches he/she earned.
     \item As a User, I want to earn rewards/batches for my engagement within Blubb.
     \item As a User, I want to edit my "thread profiles".
     \item As a User, I want to switch off notifications on my mobile.
     \item As a User, I want to change my profile picture by uploading another one.
     \item As a User, I want to change how often my mobile pulls new messages.
     \item As a User, I want to show others what I'm doing right now via some kind of "status".
     \item As a User, I want to have a profile picture which is shown with my messages.
     \item As an Admin, I want to reset a Users password when he/she forgot her password.
     \item As an Admin, I want to delete a thread from an User.
     \item As an Admin, I want to delete messages from an User.
     \item As a User, I want to have some kind of link to quoted messages so that I can read immediately what has been quoted.
     \item As a User, I want to quote another message.
     \item As a User, I want to hide different threads on my mobile and on the web front end. For this I want some kind of profiles where I can manage which thread to hide.
     \item As a User, I want to hide messages from threads so I don't have to read all the messages from e.g. the coffee break thread.
     \item As a User, I want to search for a word within a single thread.
     \item As a User, I want to search for a word and expect to see all messages from all threads containing this word, sorted chronological.
     \item As a User, I want to view my "like Count" for a period of time, e.g. last week.
     \item As a User, I want to view my "like Count" so I can see how many likes I've got for all my messages.
     \item As a User, I want to be able to "like" a good message.
     \item As a User, I want to share an audio file.
     \item As a User, I want to upload an audio-file.
     \item As a User, I want to view a uploaded/shared video within a message.
     \item As a User, I want to see uploaded/shared pictures within the message.
     \item As a User, I want to share a video in a thread.
     \item As a User, I want to upload a video to a message I'm writing.
     \item As a User, I want to share a picture in a thread by using the Share Button from Android and be able to write some text to this.
     \item As a User, I want to upload a picture to a message I'm writing.
     \item As a User, I want to change a message I wrote.
     \item As a User, I want to delete a message I wrote and nobody can read it anymore or reply to it. 
     \item As a User, I want to add descriptions to my threads.
     \item As a User, I want to open a new thread to talk with others about a specific topic.
     \item As a User, I want to get a Notification on my mobile about a new message.
     \item As a User, I want to reply to a message I'm reading.
     \item As a User, I want to write a text message for a specific thread.
     \item As a User, I want to read a message from a specific thread. 
   \end{itemize}
\section{Current state}

\section{Next steps}

\chapter{Technical basics}

\section{Beap and BeapDB}

\section{Android}

\section{Permissions and user types}

\chapter{Blubb android app}


\section{Architecture overview}
 Beap lightweight user bulletin board (\blubb{}) is an application to manage the communication within a project. A mobile front end improves the availability of the project members. They become mobile reachable with their smartphones. Unlike a mobile website a native app offers the possibility that the user can read messages even offline. Additionally it can use notifications to get the attention of users and improve their availability. \\
In order to realize this task an android application has been implemented. This technical documentation describes the way  \appname{} is operating.\\
\appname{} can be set apart in three main parts: 

\begin{itemize}
\item{The Data storage,}
\item{The View (Front end, user interface),}
\item{The Managers or controllers.}
\end{itemize}

The data storage itself consist of two nearly redundant databases. A \beapDB{} at a \beapServer{} is the central database for all user. It is the exchange server for all data. The second is a local SQLite database mirroring the data form the \beapDB{}. This is needed to provide the offline access and to add some user specific fields to the data sets, e.g. that a message has already been read.\\
Within android activity classes implement the user interface. At the current state of the project there are four different activities:

\begin{itemize}
\item{ThreadsActivity}
\item{MessagesActivity}
\item{LoginActivity}
\item{SettingsActivity}
\end{itemize}

\begin{figure}[!ht]
	\centering
    \includegraphics[width=\linewidth]{ActivitiesOverview.png}
	\caption{Activities navigation.}
\end{figure}

The main purpose of the ThreadsActivity is to show the list with available BlubbThreads but it is the center of the applications user interface, too.\\ A user can access all the other activities from there. Through a click on one list item he starts the MessageActivity. This shows a list of BlubbMessages that belong to the clicked thread. \\
- user can access login and settings through the menu of threadsActivity
- loginActivity is for reset pw too
- Settings will be accessed through the menu of ThreadActivity, too.
- From Login, Messages and Settings BackButton always goes to ActivityThreads.


- User can input data: 

at LoginActivity:
- login
- ResetPw
- Initialize pw, throgh dialog called when pw is init password

at ThreadsActivity:
- create new thread (title and description)
- set thread (title, description, status)

at MessagesActivity:
- create new message (text content, later other content types)
- edit message(text content)
- reply to message
- jump to message replied to

at settingsActivity
- set frequency of the background pull service

additionally a background service. Via alarm service of android.
performs a quickCheck at beap and compares the number of threads and messages at beap and locally at the sqlite db. if new content available makes notification to inform user.

link between views and data storage are three manager classes
\begin{itemize}
\item{SessionManager}
\item{ThreadManager}
\item{MessageManager}
\end{itemize}

sessionmanager holds sessioninfo - manages all matters with the session: performs login from loginActivity - makes auto login when username and password are at preferences - makes session refresh if necessary and provides the session id needed for all interactions with the beap server quickCheck 

ThreadManager manages all matters with blubbthreads. get threads form beap server, get threads from the sqlitedb, compare both and add new from server as 'new' marked threads to sqlite, add a thread to both, edit thread at both, get new threads

MessageManager similar to ThreadManager. get messages for thread from beap, from sqlite, compare both lists and mark new as 'new' 

\section{Threads and messages}
\subsection{Purpose}
\appname  is an application to manage the communication for small scale projects. Communication is the act of sending a message from a sender to a receiver. In case of \appname  the sender is always a single person but the receiver are all participants of the project. Even in a very small project a lot of messages must be sent to maintain a proper communication. So it is important to organize all messages in subjects, otherwise it would end in a mess.

To implement this model three objects are needed. User representing the sender and receiver, messages that can be sent and threads to organize the messages to subjects. A user is represented by the user ID. That is also the unique user name within the application. Messages are implemented by the BlubbMessage class and threads by the BlubbThread class.

\subsection{Messages}
\subsubsection{The class BlubbMessage}
An object of the class BlubbMessage represents a message within the \appname  application. Such an object holds all information about the message plus it can create a MessageView for the visual representation in an activity. 

To hold all the information a BlubbMessage has many fields, mostly of the type string: 
\begin{itemize}
\item \code{String mID}\\
Unique ID of the message.
\item \code{String mTitle}\\
Title of the message. For text messages this is optional but must be sent to the beapDB as "UNDEFINED".
\item \code{MContent mContent}\\
In the current version only TextContent is available. For further content types the interface MContent must be implemented. 
\item \code{String mCreator}\\
The unique ID/username of the user who wrote the message.
\item \code{String mCreatorRole}\\
This is mainly used to highlight special kinds of users like admins or project leader, since their content might be more important.
\item \code{String mType}\\
Through the message type it will be more easy to identify the content of the message.
\item \code{String mThread}\\
The unique ID of the thread the message belongs to.
\item \code{String mLink}\\
If the message is a reply to another message the link points to this message and contains the ID of it.
\item \code{Date mDate}\\
The date of creation of the message. 
\item \code{boolean isNew}\\
The field isNew indicates whether the user of the app has read the message. It will be set to 'true' when loading it the first time from the beapDB and to 'false' when the user closes the ActivityMessages for the messages thread (in \code{ActivityMessages.onStop()}). 

\end{itemize}

A BlubbMessage can be constructed in two ways. Either with a proper formatted JSON-Object, like in listing \ref{lst:MessageJson} or by providing all needed fields. 
\begin{lstlisting}[caption=Message JSON object, label=lst:MessageJson]
{
        "mId": "m2014-07-16_170322_S-Gross",
        "mType": "Message",
        "mCreator": "S-Gross",
        "mCreatorRole": "admin",
        "mDate": "2014-07-16T17:03:22.000+0200",
        "mThread": [
            "t2014-07-03_165629_B-Richter"
        ],
        "mTitle": "Thread bearbeiten",
        "mContent": "@B-Richter\nwerde ich checken ...",
        "mLink": "m2014-07-16_150532_B-Richter"
}

\end{lstlisting}
The MessageManager builds BlubbMessages when receiving messages from the beap server. Therefor it uses the JSON object in the response. The field 'isNew' will then automatically set to true.

The SQLite database uses the constructor where all fields must be set. 

\subsubsection{MessageView}
A MessageView represents a message in the UI and is created by a BlubbMessage object. It is shown as an item in the ListView of the ActivityMessages.

How a MessageView looks depends on three factors. A message from the thread creator will have a lighter background, messages from the current user appear on the right side of the screen the others on the left side and new messages have a red background. Combining these, results to eight different message appearances.
The basic layout for all is the \code{message_layout.xml}. The views contained in the MessageView are five TextViews, two Buttons and one custom ContentView.
The TextViews are for:
\begin{itemize}
\item \code{message_layout_icon_tv}\\
For the icon of the message. The font of this TextView is set to BeapIconic and will show either the letter 'U' or '@' if the message is a reply to another message. According to the role of the creator the color will be red for admins, yellow for project leader and blue for regular user.

\item \code{message_layout_creator_role_tv}\\
Shows the creator's role as text. Currently it is set 'INVISIBLE'.

\item \code{message_layout_creator_tv}\\
Shows the user name of the message creator.

\item \code{message_layout_title_tv}\\
This TextView displays the title of a message. The current text messages are entered without title, so this will be mostly empty but nevertheless important for further message types.

\item \code{message_layout_date_tv}\\
The date when the message was created will be displayed in this TextView. In a BlubbMessage object the date is represented by a Date object. A SimpleDateFormat object from the BlubbMessage formats the string representation of the date to a specified pattern. This is defined in the constant \code{BlubbMessage.DATE_PATTERN}.

\end{itemize}

\begin{figure}[!ht]
	\centering
    \includegraphics[width=\linewidth]{MessageView.png}
	\caption{MessageView example}
\end{figure}

\subsection{Threads}
\subsubsection{The class BlubbThread}
In a project, there are different subjects to discuss. A thread contains the messages for one subject, e.g. bugs in a software development project. This enables to give the communication in a project a certain structure. 

The class BlubbThread represents a thread in \appname . Like the BlubbMessage it has many fields to hold all information:
\begin{itemize}
\item \code{String tId}\\
Unique ID for the thread.
\item \code{String tTitle}\\
The title or name for a thread.
\item \code{String tDesc}\\
Description for the thread, that should briefly explain its purpose.
\item \code{String tCreator}\\
ID of the creator.
\item \code{String tCreatorRole}\\
Like in a message, this is mainly used to highlight special kinds of users like admins or project leader, since their content might be more important.
\item \code{ThreadStatus tStatus}\\
The status of a thread can either be OPEN, CANCELED or SOLVED. Is it CANCELED or SOLVED it is not possible to post any new messages. SOLVED hereby indicates that the subject has come to a successful end, while CANCELED indicates it has failed.
\item \code{Date tDate}\\
The date of creation of the thread.
\item \code{int tMsgCount}\\
The number of messages belonging to this thread.
\item \code{boolean isNew}\\
Indicates whether the thread is new or not.
\item \code{boolean hasNewMsgs}\\
When the thread has new messages this will be 'true' otherwise it will be 'false'.

\end{itemize}

Like the BlubbMessage a BlubbThread has two constructors. One needs every of the fields described above as parameter the other just a proper formatted JSON object. The formatting of it is shown in listing \ref{lst:ThreadJson}

\begin{lstlisting}[caption=Thread JSON object, label=lst:ThreadJson]
{
        "tId": "t2014-07-03_165629_B-Richter",
        "tType": "Thread",
        "tCreator": "B-Richter",
        "tCreatorRole": "PL",
        "tPath": [
            "Thread"
        ],
        "tDate": "2014-07-03T16:56:29.000+0200",
        "tTitle": "Bugs",
        "tDescr": "Bitte hier unerwartetes oder fehlerhaftes 		Verhalten der App eintragen. ",
        "tMsgCount": 10
    }
\end{lstlisting}
The ThreadManager builds BlubbThreads with the JSON object, when receiving threads from the beapDB. Since the fields 'isNew' and 'hasNewMsgs' are not included in the JSON object they will be set 'true'. This is because only new threads are not contained in the SQLite database and will be added instead of updated. 

The SQLite database uses the other constructor to create BlubbThread objects.

\subsubsection{ThreadView}
ThreadViews are created by BlubbThread objects. In fact there are two Views to represent a thread in the ListView of the ActivityThreads, a small and a big one. 
The small shows the most important information about the thread like the title, creator, status and the message counter. Additionally the big view shows the creator role, date, description and a button for editing the thread. The user can switch these views with a long click on the current ThreadView. 

The file \code{thread_head_layout.xml} defines the layout for the part both views use:
\begin{itemize}
\item \code{thread_status_tv}\\
Shows the status of the thread. The font for this TextView is set to BeapIconic to show the icons indicating the status. A head or letter 'U' stands for OPEN, a green check mark for SOLVED and a red x mark for CANCELED. The color of the head stands for the role of the creator, red for admins, yellow for project leader and blue for regular user. 
\begin{figure}[!ht]
	\centering
    \includegraphics[width=\linewidth]{StatusIcons.png}
	\caption{Statuses of threads.}
\end{figure}

\item \code{thread_title_tv}\\
TextView for the title.

\item \code{thread_autor_tv}\\
TextView for the creator.

\item \code{thread_msg_count_tv}\\
TextView for the message counter. If the thread has a new message the color of it will be red.

\end{itemize}
The small ThreadView is defined in the file \code{thread_small_layout.xml} and contains only the head layout. Custom layouts can be integrated in layouts with the 'include' tag, see listing \ref{AsyncCheckLogin}.

\begin{lstlisting}[caption=Thread small layout, label=lst:ThreadSmallLayout]
<?xml version="1.0" encoding="utf-8"?>
<LinearLayout xmlns:android="http://schemas.android.com/apk/res/android"
    android:layout_width="match_parent"
    android:layout_height="wrap_content"
    android:background="@drawable/threadview_rc_transpgray_bg"
    android:orientation="horizontal"
    android:paddingLeft="20dp"
    android:paddingTop="10dp">

    <include layout="@layout/thread_head_layout" />
</LinearLayout>
\end{lstlisting}

In addition to the head layout the big view contains following items:
\begin{itemize}
\item \code{thread_info_tv}\\
Shows the creator role and the date of creation for the thread.

\item \code{thread_description_tv}\\
Shows the description of the thread.

\item \code{thread_edit_button}\\
The button to open the edit thread dialog. If the current user is not the creator of the thread it will be invisible. The font of it is the BeapIconic font and shows the letter 'E'.

\end{itemize}

\begin{figure}[!ht]
	\centering
    \includegraphics[width=\linewidth]{ThreadView.png}
	\caption{ThreadView example}
\end{figure}

\section{Data storage}

\subsection{beapDB at the beap server}

\subsubsection{Communication with the beap server}
Communication to the beap server via http requests 
in manifest request permission for network - all actions demanding network work need to be in a separate thread not on user thread, see AsyncTasks.
general scheme
	- build a request string with RequestBuilder
	- execute the html request with htmlRequest
	- receive the response string and build a BlubbResopnse object
	- pass this object to the calling method
	- extract the result object within the BlubbResponse according to what requested, e.g. tree.functions.getAllThreads(self) build the BlubbThreads from the jsonArray that is the result.
	
\subsubsection{Building request strings}
two different kinds of request strings, one for the beap server (all about the session) the other are queries at the beapDB, therefor they have different beapids..
A request string for session contains following parts:
 	- the url: http://blubb.traumtgerade.de:9980/
 	- question mark for request: ?
 	- BeapId: BeapSession 
 	- Action: login, refresh, logout, check or setOwnPwd 
 	- beap app version: at login
 	- parameter: for session
 		- login: uName, uPwd
 		- check: BeapSession
 		- refresh: BeapSession
 		- logout: BeapSession
 		- resetOwnPwd: uName, uPwd, newPwd1, newPwd2
A request string for a login would look like this:
\url{http://blubb.traeumtgerade.de:9980/?BeapId=BeapSession&Action=login&appVers=1.5.0 rc1&uName=Der-Praktikant&uPwd=test}
request for beapdb is different:
	- url:
	- question mark:
	- BeapId: beapDB
	- sessionId:
	- action: query
	- queryStr: tree.functions.getAllThreads(self)
parameter for the query in the braces, first parameter is a reference to the session object. for this application it's always the own session, so it's always self.
Following a list of custom queries/functions available prefix always "tree.functions.:
\begin{itemize}
\item{\code{getAllThreads(self)}}
\item{\code{createThread(self, "Thread title", "Thread description")}}
\item{\code{setThread(self, "Thread id", "Thread title"[opt.], "Thread description"[opt.], "Thread status"[opt.])}}
\item{\code{getMsgsForThread(self, "Thread id")}}
\item{\code{createMsg(self, "Thread id", "Message title"[opt.],MessageContent, "Message link"[opt.])}}
\item{\code{setMsg(self, "Message id", "Message title"[opt.], "Message content"[opt.], "Message link"[opt.])}}
\item{\code{quickCheck(self)}}
\end{itemize}


 \subsubsection{Parsing parameter}
every parameter send to the beapDB like username, password, a text content of a message etc. need to be encoded to UTF-8.
BPC (Blubb Parameter Checker) uses a URLEncoder to encode the parameter to utf-8. 
Additionally escape character must be escaped like:
	- newline "\textbackslash n" to "\textbackslash \textbackslash n"
	- tab "\textbackslash t" to "\textbackslash \textbackslash t"
	- "\textbackslash \textbackslash \textbackslash ""
When all newline, tabs and " are escaped the whole string is packed within two " and then encoded to utf-8.
Lets try a simple example. A user types a message like:

\begin{lstlisting}[caption=Example string]
We need to encode character like:
	"`\textcolor{mauve}{ü}`"
\end{lstlisting}
\label{ExampleString}

First the newlines, tabs and quotation marks must be escapted. The resulting string would look like this:

\begin{lstlisting}[caption=Escaped character]
"We need to encode character like:\n\t\"`\textcolor{mauve}{ü}`\""
\end{lstlisting}
\label{ExampleStringExcaped}
After this the string is encoded in utf-8 and will be send to the server like:

\begin{lstlisting}[caption=UTF-8 encoded string]
"We%20need%20to%20encode%20character%20like%3A%5Cn%5Ct%5C%22%C3%BC%5C%22"
\end{lstlisting}
\label{ExampleStringUTF8}

The parameter send to the Request manager for all session request are parsed automatically. Parameter entered in a database query must be parsed with BPC by calling the static method parseStringParameterToDB, just like in the following example.
\begin{lstlisting}[caption=Parse a string parameter]
	...
	String result = BPC.parseStringParameterToDB("\n`\textcolor{mauve}{ü}`bercool");
	...
\end{lstlisting}
\label{ParseStringParameter}

\subsubsection{Response from beap}
The response from beap will always be a json object with the following structure: 
\lstset{language=JavaScript}
\begin{lstlisting}[caption=Json object for beap response]
{
		BeapStatus : <int>, 	/* Response status, e.g. 200 for OK.*/
		StatusDescr : <string>, /* A description for the status.*/
		Result : <var>,         /* If available a json object a result.*/
		sessInfo : {			/* Info object about the session.*/
			sessId : <MD5-string>,
			sessUser : <string>,
			sessRole: <string>,
			sessActive : <bool>,
			expires: <GMT-Date>  
		}
	}
\end{lstlisting}
\label{beapResponse}
\lstset{language=java}
BeapStatus and StatusDescr describe the status of a request at beap. The following are the most common statuses:
\begin{itemize}
\item{200 - OK}
\\The request could be processed properly and will contain the expected result.
\item{203 - No content}
\\The request could also be processed properly but there is no result.
\item{204 - session already deleted}
\\The sessionId is not valid any more.
\item{400 - request failure}
\\A response will have this status mostly if parameter are missing.
\item{401 - login required}
\\If a login is required before sending this request.
\item{403 - permission denied}
\\This status will be send if a user has not the permission to request a certain action at the beapDB.
\item{407 - connection error}
\\If there is no network connection or the server is offline the response object will have this status.
\item{406 and 409 - parameter error}
\\In the request was something wrong with a parameter, e.g. a string has not been encoded properly.
\item{418 - syntax or reference error}
\\The status for a error within a query string.
\end{itemize}
The result object will vary due to the request. Integer, json object or json array are the possible types of the result. 
For details see appendix A.

\subsection{SQLite as a local database}

For android there are different storage options\citep{aDefStorageOpt}:
\begin{itemize}
\item{shared Preferences \\key value pairs}
\item{internal storage \\private data for app e.g in xml}
\item{external storage \\public data on shared external storage}
\item{SQLite databases \\Store structured data in a private database}
\item{network connection \\Store data with own network server.}
\end{itemize}
 

For blubb used shared preferences for username, password, settings of pull service, for quickcheck the number of threads and messages at beapdb to compare them and request if new are available.

BlubbThreads and BlubbMessages are stored with a SQLite database. 
Class accessing the database is the DatabaseHandler.
it implements the SQLiteOpenHelper, which manages database creation and version management \citep{aDefSQLiteOpenHelper}

The version number is a constant within the class and must be increased if any changes are made at the database, e.g. adding a new field to a table - if not database will probably crash.
the names for the database, tables and column names are set with constants too. They should be accessed only through these constants to avoid typos. 

Database consists of two tables - threads and messages
they are created in onCreate(), it's called the first time the database is created. The primary key for the threads are the thread id, for the messages the message id, since they are already unique from the beapDB. The tables are connected via the thread id which is given for every message.
The tables are nearly the same as at beap but have columns for the boolean values isNew in messages, isNew and hasNewMsg in threads. The boolean values are stored as ints and must be parsed. <1 equals true and 0 equals false. 

\begin{figure}[!ht]
	\centering
    \includegraphics[width=\linewidth]{BlubbERM.png}
	\caption{ERM of the SQLite database.}
\end{figure}


plain sqlstatement for the creation or threads and messages:
\lstset{language=SQL}
\begin{lstlisting}[caption=SQL code for creating the table messages]
CREATE TABLE messages(
	mId TEXT PRIMARY KEY,
	mTitle TEXT,
	mContent TEXT,
	mRole TEXT,
	mCreator TEXT,
	mDate TEXT,
	mType TEXT,
	mThreadId TEXT,
	mLink TEXT,
	mIsNew INTEGER)
\end{lstlisting}
\label{SQLTableMessages}
\begin{lstlisting}[caption=SQL code for creating the table threads]
CREATE TABLE threads (
	tId TEXT PRIMARY KEY,
	tTitle TEXT,
	tDesc TEXT,
	tCreator TEXT,
	tCreatorRole TEXT,
	tStatus TEXT,
	tDate TEXT, 
	tMsgCount INTEGER,
	tType TEXT,
	tIsNew INTEGER,
	tHasNewMsg INTEGER)
\end{lstlisting}
\label{SQLTableThreads}
\lstset{language=Java}

The databaseHandler offers some methods to access the database:

\begin{itemize}
\item{\code{addMessage(BlubbMessage message)} 
\\Adds a message to the database.}
\item{\code{addThread(BlubbThread thread)}
\\Adds a thread to the database.}
\item{\code{getMessage(String mId)}
\\Gets a message with the provided message id.}
\item{\code{getThread(String tId)}
\\Gets a thread from the database.}
\item{\code{getAllMessages()}
\\Gets all messages stored in the database. (not used yet)}
\item{\code{getMessagesForThread(String tId)}
\\Gets all messages for a thread.}
\item{\code{getAllThreads()}
\\Gets all threads stored in the database.}
\item{\code{setMessageRead(String mId)}
\\Sets the read flag of a message to true.}
\item{\code{setThreadNewMsgs(String tId, boolean hasNewMsgs)}
\\Sets the 'hasNewMsgs' flag of a thread.}
\item{\code{updateMessage(BlubbMessage message)}
\\Updates the values of a stored message, e.g. when the content has changed.}
\item{\code{updateMessageFromBeap(BlubbMessage message)}
\\This updates without the isNew flag because this can not be modified at the beapDB.}
\item{\code{updateThread(BlubbThread thread)}
\\Update the values of a stored BlubbThread, e.g. when the description has been changed.}
\item{\code{updateThreadFromBeap(BlubbThread thread)}
\\This updates a thread without changing the 'isNew' or 'hasNewMsgs'.}
\item{\code{getMessageCount()}
\\Gets the number of messages stored in the database.}
\item{\code{etThreadCount()}
\\Gets the number of threads stored in the database.}
\end{itemize}

The database can easily be accessed from any class of the project. Just instantiate an instance and call the desired method like in example. In order to make an instance the applications context must be provided, due to that the SQLite database is only accessible for the application which created it.
\begin{lstlisting}[caption=Accessing the SQLite database]
	...
	BlubbMessage message = new BlubbMessage(result);
	DatabaseHandler db = new DatabaseHandler(context);
	db.addMessage(message);
	...
\end{lstlisting}
\label{SQLiteAccessExample}

\section{Manager}

\subsection{Singleton pattern}
used the singleton pattern for all manager classes to provide global access to them. it's realized by making the constructor of the classes private and create a static instance. 
\citep[p. 39]{cooper2000java}
- one problem with android. singletons must be created within a context that's every time available.
- Therefor the e.g. MessageManager.getInstance() must be called within the application class one time. 
- android does not allow singltons to be instantiated other ways 
- Extended the application class with BlubbApplication
- the application can be accessed in activities an services via .getApplicaiton() and parsing the result to a BlubbApplication
in the android manifest file need a entry for the used application class see example:

\lstset{language=xml}
\begin{lstlisting}[caption=Application tag of the manifest.xml]
...
<application
        android:name=".blubbbasics.BlubbApplication"
        android:allowBackup="true"
        android:icon="@drawable/blubb_logo"
        android:label="@string/app_name"
        android:theme="@style/AppTheme">
        ...
\end{lstlisting}
\label{lst:AppTagManifest}

\lstset{language=java}
pay attention when making subclasses of singletons
\\subclassing a singleton can be difficult, since this can work only if the base Singleton class has not been instantiated. \citep[p. 46]{cooper2000java} 


\subsection{SessionManager}
All queries at the beapDB need a valid session Id to authenticate the users identity and check weather the user has the permission to perform a certain query. after a login the beap server returns a response object. One part of this object is a SessionInfo object. The SessionManager holds this SessionInfo to provide access to it and the session Id within it.
if the user has saved the credentials the login will be performed automatically, if not a SessionException will be thrown and the user should be asked to log in. The username and password will be stored temporarily so the user must not be bothered again while the app is running. 
If the password is "init" the response status will be passwordInit, which means the user must set an own password and should be asked to do so. \\

The session manager also provides a method to do a quickCheck at the beapDB and see whether there are any new threads or messages available. Beap will return a result array with just to integer. The first is the number of threads, the second the number of messages at the beapDB. If there are new threads or messages quickCheck will request them and return a QuickCheck object containing two lists with the new threads and messages.
Furthermore the logout and password reset can be done through the SessionManager.
Following a complete list of methods provided by the SessionManager:

\begin{itemize}
\item{\code{getSessionID(Context context)}}\\
Get the session ID. If necessary and possible this will perform a auto login or session refresh.

\item{\code{login(String username, String password)}}\\
Make a manual login. After this the session Id will be available and the session will be refreshed automatically.

\item{\code{quickCheck(Context context)}}\\
Make a quickCheck at beap and see whether new threads or messages available. If so the new ones will be available at the QuickCheck object like: \begin{lstlisting}
QuickCheck quickCheck = sessionManager.quickCheck(context);
List<BlubbThread> threads = quickCheck.threads;
List<BlubbMessage> messages = quickCheck.messages;
\end{lstlisting}

\item{\code{resetPassword(String username, String oldPassword, String newPassword, String confirmNewPassword)}}\\
Performs a tree.functions.resetPwd(...) at beap and will return true if successful.

\item{\code{getUserId(Context context)}}\\
Returns a string with the current users username.

\item{\code{logout(Context context)}}\\
Logs the user out of the beap server, the session will not be active any more and a formerly, at the shared preferences saved password will be deleted. After this the user needs to log in manually.


\end{itemize}

\subsection{ThreadManager}
The Thread manager manages the access to all available threads. When it requests the threads from the beapDB it updates the SQLite database immediately. 

\begin{itemize}
\item{\code{getAllThreads(Context context)}}\\
Gets first the threads from beapDB and updates the SQLite database then returns the list of threads from the SQLite database.

\item{\code{getNewThreads(Context context)}}\\
Get a list with all threads with the isNew tag set to true.
 
\item{\code{updateAllThreadsFromBeap(Context context)}}\\
Update all threads at the SQLite database from the beapDB. Load them from SQLite.

\item{\code{getAllThreadsFromSqlite(Context context)}}\\
Get all the threads stored on the local SQLite database.

\item{\code{getThreadFromSqlite(Context context, String tId)}}\\
Get a single thread from SQLite database.

\item{\code{createThread(Context context, String tTitle, String tDescription)}}\\
Creates a new Thread at beap. If everything goes well the thread will be added to the SQLite database and the whole new BlubbThread object will be returned.

\item{\code{readingThread(Context context, String threadId)}}\\
Call this if a thread is shown on the user interface and the thread is not longer a new one.

\item{\code{setThread(Context context, BlubbThread thread)}}\\
Updates the thread at beapDB and the sqlite database.

\end{itemize}


\subsection{MessageManager}
Like the ThreadManager the MessageManager manages all concerns about the messages. It is the link between the data storage, local and at the beap server and the user interface. Through the following methods it can provide this service.
\begin{itemize}
\item{\code{getNewMessagesFromAllThreads(Context context)}}\\
Get all messages from all threads where isNew is true. The message count of all local and all threads on beapDB are compared, when there is a difference on a thread getNewMsgsForThread(..) will be called and all new messages for a thread added to the returning list.

\item{\code{createMsg(Context context, String... messageParameter)}}\\
 Creates a new message on the beapDB, on a positive response the message will be added to the SQLite database.
     
\item{\code{getAllMessagesForThread(Context context, String tId)}}\\
Get all messages for a thread.

\item{\code{getAllMessagesForThreadFromBeap(Context context, String tId)}}\\
Get all messages for a thread from beapDB.

\item{\code{putMessageToSqliteFromBeap(Context context, BlubbMessage message)}}\\
 Put a message to the SQLite database. If the db contains a message with the same id it will be updated otherwise it will be added.
     
\item{\code{getAllMessagesForThreadFromSqlite(Context context, String tId)}}\\
Get all messages for a thread, stored in the local SQLite database.

\item{\code{setMsg(Context context, BlubbMessage message)}}\\
Changes a message at beapDB and the local SQLite database.
\end{itemize}

\section{View or Front end}

\subsection{Basics}
In android the visualization made in activities. They provide a screen with which users can interact in order to do something\citep{aDefActivities}.\\
 Different layout types provide different structures for the screen layout, e.g. a LinearLayout orders its'  children in a linear structure, either vertically or horizontally. A RelativeLayout for example will order them relatively to each other. \\
 The children of these layouts can be even layouts or views. Android provides many Views. TextViews to display text, ImageViews for images, Buttons and many more. The Views are Items at the screen the user actually sees and interacts with.\\
The user interface of \appname mostly uses TextViews, EditTexts and Buttons. TextViews only show text. The user can not interact with the text. Therefore a special kind of TextView, the EditText is made for. Buttons also are just a specialized TextView, mostly different in its appearance. \\
It is possible to define the user interface programmatically or with XML files. The android framework provides the possibility to use either or both. For \appname mainly the xml files came to use. 
When developing an application for android it is important to consider the activities life cycles. See figure~\ref{fig:activitylivecycle}.
 
\begin{figure}[!ht]
    \includegraphics[width=\linewidth]{activity_lifecycle.png}
	\caption{Application lifecycle.} \floatfoot{Source: \citep{aDefActLCPic}}
	\label{fig:activitylivecycle}
\end{figure}

Usually in \code{onCreate()} the content view for the activity is set, that is, the UI layout is placed in the window of the activity. Other actions, e.g. saving the current state of the application, should be considered at the appropriate life time events.
\subsection{ActivityThreads}
At the start of the application the first activity is the ActivityThreads.
purpose is to display all available threads. Three different thread types , admin thread, PL thread and user thread, see permissions.
Regular view of threads just status and type, title, author and msgcounter. Extended view shows additionally date, description and if the current user has the permission the edit button.
The user can perform following actions at the ActivityThreads:

\begin{itemize}
\item{Create a new thread.}\\
By clicking on 'New thread' in the menu the user opens a dialog where he can enter the title and description and create a new thread.

\item{Refresh the list of threads.}\\
This option in the menu will reload the threads from the beap server and update the SQLite database and the ListView with threads.

\item{Go to the Settings screen.}\\
The ActivitySettings will start when the user clicked 'Settings' in the menu.

\item{Go to the Login screen.}\\
In case the user needs to manually log in he can start the ActivityLogin with the LoginType LOGIN, by clicking 'Login' in the menu. 

\item{Go to the reset password screen.}\\
To reset the password the user opens the menu and clicks 'Reset password'. This will start the ActivityLogin with the LoginType RESET.

\item{Log out.}\\
If the user wants to log out he clicks 'Logout' in the menu. In case the password was stored in the Preferences it will be deleted. The Pull service will be stopped and the Application will be send to the background. Unfortunately it's not possible to close an application. This is reserved to the OS.

\item{Toggle the size of  a thread view.}\\
A long click at one thread view will toggle it's size between the big view and the small one.

\item{Modify a thread.}\\
The user can modify his own threads by clicking at the 'EditButton', with the pencil icon. This will open a dialog window where the title, description and status can be changed.

\item{Go to the messages of a thread.}
A click on a thread within the list opens the ActivityMessages for this thread. 

\item{Scroll through the list to see all threads.}\\
If the list has more than 7 items the user needs to scroll to see all entries.

\end{itemize} 

\subsubsection{Start of the activity}
When the ActivityThreads is started it will first check whether it's possible to log in at the beap server. If the username and password is not stored at the Shared Preferences it will check whether threads are available at the local SQLite database. In case neither a login is possible nor any threads are stored the ActivityLogin will be started. Otherwise the BlubbThreads are loaded form the SQLite database and displayed in the ListView of the activity. But if a automatic login is possible the BlubbThreads will be updated form beap and reload to the list.

\subsubsection{Layout structure}
The Layout for the ActivityThreads is defined in \code{threads_activity_layout.xml}.
this just defines the ListView for the threads, and a progressBar to show the loading process when a network action is performed. For a thread within the ListView two different Views are available. The small one (\code{thread_small_layout.xml}) shows only an icon, the title of the thread, who created it and how much messages it contains. Additionally the big view (\code{thread_big_layout.xml}) shows a date when the thread has been created, the role of the creator and the description. If the current user is the creator of this thread, a button will be shown, too. With it the user can open the edit thread dialog. \\
Two dialog windows are also part of the ActivityThreads. With the create thread dialog (\code{create_thread_dialog.xml}) the user can open a thread for a new subject. Therefor he must enter the title and a description. To change this, e.g. because of a typo, the user can open the edit thread dialog (\code{edit_thread_dialog.xml}) and change it. With this he also can change the status of a thread. 

\begin{figure}[!ht]
	\centering
    \includegraphics[width=\linewidth]{ActivityThreadsLayouts.png}
	\caption{Layouts for the ActivityThrads.}
\end{figure}


\subsubsection{Menu}
Most actions are accessed through the menu. The menu is a common user interface component, to provide a familiar and consistent user experience the Menu API is used to present actions and other options in activities\citep{aDefMenu}. Some devices have even a 'hard' button to access it. The menu itself is defined in a xml file. The menu for ActivityThreads is defined in \code{activity_threads_menu.xml} in the res/menu folder.
Most actions a user can perform with this activity are accessed through the menu. The actions are defined in the \code{ActivityThreads.class} within the \code{onOptionsItemSelected()} method.

\subsubsection{ListView, ArrayAdapter and ThreadViews}
A ListView is a view group that displays a list of scrollable items\citep{aDefListView}. The items are loaded to the ListView via a ListAdapter. In case of the ActivityThreads, the ListAdapter is an ThreadArrayAdapter, extending a ArrayAdapter, containing a list of BlubbThread objects. The items for the ListView are provided by these objects. Should a object of its list be displayed the ThreadArrayAdapter calls \code{getView()} on it. Depending on the current state of the thread it will return a small or a big view. The threads have an OnClickListener and an OnLongClickListener, both given to it in the constructor of the ThreadArrayAdapter. On click on a thread view a new Intent will be constructed and the thread ID will be given as an extra to it. The Intent will start a ActivityMessages for the clicked thread.\\
On a long click the view will change its size or switch from its layout between \code{thread_small_layout.xml} and \code{thread_big_layout.xml}. A button on the big layout will start the edit thread dialog.\\

\subsubsection{Create thread and edit thread dialogues}
To create a new thread the user has to enter the title and a description in a dialog window. The layout for the create thread dialog is defined in \code{create_thread_dialog.xml}. It consists of two LinearLayouts. The first contains a TextView for the dialog title, two EditTexts for entering the title and description for the thread. Two buttons are in the second. The green one to create with the entries of the two EditTexts a new thread the other to cancel the dialog window. When the user clicks the green button a new AsyncNewThread will be created and executed. See \ref{subsubsec:AsyncNewThread}. To start a create thread dialog \code{newThreadDialog()} must be called from within an ActivityThreads object.
The dialog to modify a thread is similar build like the dialog to create a new thread. Except that between the two buttons lays a Spinner containing the different statuses for the thread. A Spinner is an android equivalent for a drop-down list or combo box. The user can select one of three predefined items. For the status they are:
\begin{itemize}
\item OPEN
\item SOLVED
\item CANCELED
\end{itemize}
If the user hits the green button on this dialog an AsyncSetThread will be executed. For details see section \ref{subsubsec:AsyncSetThread}. The dialog window will start if \code{editThreadDialog()} is called from within an ActivityThreads object.

\subsection{ActivityMessages}
The ActivityMessages will start when the user clicks on one thread in the ActivityThread. It's purpose is to show all messages belonging to the clicked thread. The appearance of a message depends on who has created it and whether it is new or has been loaded previously. The user can perform different actions in this activity:
\begin{itemize}
\item Write a new message.\\
Clicking in the EditText on the bottom of the activity will open the soft keyboard. Alternatively this can also be achieved by clicking 'New message' in the menu of the activity. After entering a text the message can be sent by clicking the green button on the right side of the EditText.

\item Refresh the list with messages.\\
This action can be accessed through the menu. It will reload the messages from beap and if new messages are available show them on top of the ListView.

\item Reply to a message.\\
Every message has a small button to reply to it. Like when writing a new message this starts the soft keyboard and works the same way as writing a new message but will be send as a reply.

\item Edit a message.\\
The user can modify own messages by clicking the edit button besides the reply button. The current content of the message will be set to the EditText and be editable for the user. Clicking the green button, again will send the content of the EditText. 

\end{itemize}

\subsubsection{Start of the ActivityMessages}
It is important to add a thread ID as an extra to the Intent that starts the ActivityMessages. With this ID the messages can be requested form the SQLite database and the beapDB. On start of the activity it will first prepare the message ListView with a header containing the thread's title, creator and description. The layout for the header is defined in \code{messages_activity_lv_header.xml}. It will also start the InputView, so the user can enter messages. After that the messages are loaded from the SQLite database and then from the beapDB. In this way the user sees most messages immediately and the new just pop up. 

\subsubsection{Layout structure}
The file \code{activity_messages_layout.xml} defines the layout for the ActivityMessages. Like the ActivityThreads it contains a ListView and a ProgressBar. The ListView shows the messages and the Progressbar indicates that a background process is in progress. On the bottom of the activity's window the InputView supplements the layout. Here can the user enter and send new messages. It consists of an EditText and a Button. For instance a new message can be entered to the EditText and be sent by clicking the Button. 
The InputView is used to write a new, modify or reply to an existing message. The different modes are set up by the methods \code{startInputView()}, \code{ replyToMessage()} and \code{changeMessage()} in the ActivityMessages.
From the menu the user can refresh the messages and start the creating of a new one. \\



The ListView of the ActivityMessage is populated with MessageViews through a MessageArrayAdapter inheriting from the ArrayAdapter class. 
In its constructor it expects a list of BlubbMessages. To get the view for a message the MessageArrayAdapter just invokes \code{getView()} on the corresponding object of this list. 

\subsubsection{Menu}
The menu at the ActionBar of a ActivityMessages has only two items, one to set up the InputView for creating a new message, the other just to reload the messages from the beap server. Clicking the item 'Create a new message' will simply call \code{startInputView()}, which sets up the InputView properly. 'Refresh' executes an AsyncGetAllMessagesToThread, for details see section  \ref{subsubsec:AsyncGetAllMessagesToThread}.

\subsection{ActivityLogin}
Only registered user can access the beapDB. Therefor they get a username and a password. With these can the user perform a login in the ActivityLogin. This activity has two different modes. One for a regular login the other to reset the password. The mode is triggered by an extra contained in the starting Intent. This extra is a LoginType and can be accessed like shown in listing \ref{lst:LoginType}. The constant \code{EXTRA_LOGIN_TYPE} holds the name needed to access the extra in the Intent.
\lstset{language=java}
\begin{lstlisting}[caption=Accessing the LoginType., label=lst:LoginType]
	Intent intent = getIntent();
    LoginType loginType = (LoginType)intent.getSerializableExtra(EXTRA_LOGIN_TYPE);
\end{lstlisting}

LOGIN is the standard LoginType. If the user enters a valid username and password and clicks the LoginButton he will be send to the ActivityThreads. Was the AcitvityLogin started with the LoginType RESET, the user can enter the username, password, new password and the new password again to detect typos early. Clicking the LoginButton will then reset the password and redirect to the ActivityLogin for a regular login. 

The very first login is a special case. At this point the user has not yet chosen an own password. 'init' is the standard password and must be reset. If the user sends a login with the init password a dialog window will pop up in which the password must be reset. 

\subsubsection{Layout structure}
Compared to the ActivityThreads or ActivityMessages the layout of ActivityLogin is pretty simple. It contains a ImageView for the logo, four EditTexts for the username and passwords, a Button and a CheckBox.

\begin{itemize}
\item \code{login_activity_logo_iv}\\
ImageView which shows the logo and name for the application.

\item \code{login_activity_username_et}\\
The EditText where the user must enter the username.

\item \code{login_activity_password_et}\\
The EditText for the current password.

\item \code{login_activity_password_reset_et}\\
EditText for a new password, when resetting it. Only visible when the LoginType is LOGIN.

\item \code{login_acitivity_password_reset_confirm_et}\\
In this EditText the user must enter the same as in \code{login_activity_password_reset_et} to avoid typos. It will also be visible only when the LoginType is RESET.

\item \code{login_activity_login_btn}\\
The text on the LoginButton depends on the LoginType. For LOGIN it will be 'Sign in', for RESET 'Reset password'. In LOGIN mode it will start an AsyncLogin, in RESET an AsyncResetPassword. See section \ref{subsec:AsyncLogin} for details to this AsyncTasks.

\item \code{login_activity_stayloggedin_cb}\\
If this CheckBox is checked when the user clicks the LoginButton, the username and password will be stored in the shared preferences. This enables an automated login, so the user must not be bothered on every start of the application.
\end{itemize}

The file \code{login_activity_layout.xml} contains the layout for the LoginActivity. 

The InitDialog is defined in the file \code{password_init_dialog.xml}. Like the ActivityLogin it contains 4 EditTexts for username, password, new password and the confirmation of the new password. Since the user already has entered the username and the init password this fields will be filled in advance. The green button will start a AsyncResetPassword, the red will cancel the dialog. 

\subsection{Notifications}
A notification is a message you can display to the user outside of your application's normal UI. When you tell the system to issue a notification, it first appears as an icon in the notification area. To see the details of the notification, the user opens the notification drawer. Both the notification area and the notification drawer are system-controlled areas that the user can view at any time \citep{aDefNotifications}.
For \appname  the notifications are used to show new messages or new threads. A background service, the PullService, is responsible for publishing these notifications. It is started by an AlarmService provided form the android system and only starts an AsyncQuickCheck. This performs a quick check at the SessionManager. If the returned QuickCheck object contains any results they will be published as notifications, like in figure \ref{fig:notifications}

\begin{figure}[!ht]
    \includegraphics[width=\linewidth]{Notification.png}
	\caption{Notifications from \appname} 
	\label{fig:notifications}
\end{figure}

A click on a notification in the notification drawer will open either the ActivityThreads or the ActivityMessages. If there is one new message the ActivityMessages will open for the corresponding thread. For more than one message or a thread notification the ActivityThreads will be started.

The PullService only can perform a quick check when the user has accepted to stay logged in in the ActivityLogin otherwise the mandatory login is not possible.

\section{Asynchronous tasks}
\subsubsection{AsyncTasks}
Tasks like network operations should not be executed on the UI thread (in this subsection a thread means the java.lang.Thread). They probably take some seconds to finish and would block the user interface meanwhile. So it is recommended to run long lasting jobs on a background thread. 
Android provides different mechanisms for running a task in the background.
Services are meant for long lasting tasks and are not affected by the lifecycle events of an activity. But they can't interact directly with the UI. \\
\appname  is mostly considered to display content provided by the beap server. Therefor many requests are send by the UI. For instance the creation of a new thread. A service could execute the request but the updating of the user interface would be tricky. The abstract class AsyncTask enables proper and easy use of the UI thread. This class allows to perform background operations and publish results on the UI thread without having to manipulate threads or handlers\citep{aDefAsyncTask}
There are two rules for background threads
\citep{aDefProcThreads}:
\begin{enumerate}
	\item Do not block the UI thread
	\item Do not access the Android UI toolkit from outside the UI thread
\end{enumerate}
With AsyncTask both can be met easily. Therefor all network operations at \appname  run on them. All AsyncTasks are inner classes of an activity. They extend the class AsyncTask and must implement the \code{doInBackgroud()} method, which will run on the background thread. The result of it is delivered to \code{onPostExecute()}. Since this runs on the UI thread, it is easy to publish the results. In the following section all AsyncTasks are explained.

\subsubsection{Exceptions in AsyncTasks}
AsyncTasks miss a proper way of exception handling. There is no build in mechanism to transfer exceptions form a background to the main thread. To maintain this matter the asyncTask classes have all a field for a BlubbException. All custom exeptions in \appname  inherit from BlubbException. In case a excetpion is thrown in \code{doInBackgroud}, it must be caught and set to this field. The handling of it can then take place in \code{onPostExecute}, see listing \ref{lst:Exception}.

\begin{lstlisting}[caption=Exception handling in AsyncTasks, label=lst:Exception]
private class AsyncNewThread extends AsyncTask<String, Void, BlubbThread> {
        
        BlubbException blubbException;

        @Override
        protected BlubbThread doInBackground(String... parameter) {
            String title = parameter[0];
            String description = parameter[1];
            try {
            	ThreadManager manager = ThreadManager.getInstance();
            	BlubbThread thread = manager.createThread(
                        ActivityThreads.this, title, description);
                return thread;
            } catch (BlubbException e) {
                this.blubbException = e;
            }
            return null;
        }
        
        @Override
        protected void onPostExecute(BlubbThread thread) {
            getApp().handleException(blubbException);
        	...
        }
	}
\end{lstlisting}
Mostly the exceptions are forwarded to the BlubbApplication. This will show a Toast message according to which type of exception it was. The text for each message is defined in the file \code{res/values/strings.xml}.

\subsection{AsyncTasks of the ActivityThreads}

\subsubsection{AsyncUpdateThreads}
\label{AsyncUpdateThreads}

If the threads need to be updated from the beapDB a AsyncUpdateThreads will call \code{updateAllThreadsFromBeap(..)} at the ThreadManager. The threads table at the SQLite database will then be updated with the threads from the beapDB. When they have been updated the ListView in the ThreadsActivity gets all threads from the SQLite database to display them.
Since the AsyncUpdateThreads needs no parameter it can simply been instantiated and executed, like shown in listing \ref{lst:AsyncUpdateThreads}

\begin{lstlisting}[caption=AsyncUpdateThreads, label=lst:AsyncUpdateThreads]
	...
	AsyncUpdateThreads asyncUpdateThreads = new AsyncUpdateThreads();
	asyncUpdateThreads.execute();
	...
\end{lstlisting}

\subsubsection{AsyncCheckLogin}
\label{AsyncCheckLogin}

At the start of the application it is necessary to check whether it is possible to log in automatically.
Therefor an AsyncCheckLogin is executed. It does not need any parameter since it tries to log in with the saved username and password in the Shared Preferences. 
The task is executed by calling \code{getSessionID(..)} at the SessionManager.


If it is possible to log in the threads will be updated and reload to the thread ListView. If not, the thread list is loaded immediately from the SQLite database. In case the thread list from the database is empty, the user logs in the first time and the ActivityLogin will be started.

\subsubsection{AsyncNewThread}
\label{subsubsec:AsyncNewThread}
A new thread must have a title and a description. Therefor the AsyncNewThread receives to string arguments in its \code{execute()} method. Like shown in listing \ref{lst:AsyncNewThread} the first is the title the second the description. 
\lstset{language=java}
\begin{lstlisting}[caption=AsyncNewThread example, label=lst:AsyncNewThread]
	...
	String title = "Bugs";
	String desription = "Please report bugs in this thread.";
	AsyncNewThread asyncNewThread = new AsyncNewThread();
	asyncNewThread.execute(title, description);
	...
\end{lstlisting}
The creation of the new thread will be made by the ThreadManager with the method \code{createThread(...)}.


After finishing the execution the application will display a short message to inform the user about the successfully accomplished task. The ListView for the threads will be updated so the new thread is shown.

\subsubsection{AsyncSetThread}
\label{subsubsec:AsyncSetThread}
To modify a thread a AsyncSetThread is needed. It receives a BlubbThread object, with the new title, description or status in its constructor, see listing \ref{lst:AsyncSetThread}.

\begin{lstlisting}[caption=AsyncSetThread example, label=lst:AsyncSetThread]
	...
	BlubbThread thread = bugThread;
	thread.setTitle("Bugs and complaints");
	AsyncSetThread asyncSetThread = new AsyncSetThread(thread);
	asyncSetThread.execute();
	...
\end{lstlisting}
Like the creation of a thread the modification will be executed by the ThreadManager. The AsyncSetThread simply calls \code{setThread(..)} on an instance of it.

If the modification of the thread was successful a Toast informs the user about it and the currently displayed threads will be updated.

\subsection{AsyncTasks of the AcitvityMessages}

\subsubsection{AsyncGetAllMessagesToThread} \label{subsubsec:AsyncGetAllMessagesToThread}
With a AsyncGetAllMessagesToThread the messages can be updated from the beapDB and shown on the UI in the ActivityMessages. In the background thread it simply calls \code{getAllMessagesForThread()} on the MessageManager. A string containing the thread ID is the only argument needed and is given to the AsyncTask in the constructor. Listing \ref{lst:AsyncGetAllMsgs} shows an example, how to use an AsyncGetAllMessagesToThread.
\begin{lstlisting}[caption=AsyncGetAllMessagesToThread example, label=lst:AsyncGetAllMsgs]
	...
	String threadID = "t2014-07-03_165629_B-Richter";
	AsyncGetAllMessagesToThread asyncTask = new AsyncGetAllMessagesToThread(threadID);
	asyncTask.execute();
	...
\end{lstlisting}

After executing the task it will update the MessageArrayAdapter in the ActivityMessages.

\subsubsection{AsyncSendMessage}
An AsyncSendMessage is used to create new messages. The arguments therefor are Strings with the thread ID, the message title, content and if it is a reply a message ID. The listing \ref{lst:AsyncSendMsg} shows an example how to use an AsyncSendMessage.

\begin{lstlisting}[caption=AsyncSendMessage example, label=lst:AsyncSendMsg]
	...
	AsyncSendMessage asyncTask = new AsyncSendMessage();
	String tID = "t2014-07-03_165629_B-Richter";
	String title = "A new bug";
	String content = "I have found a new bug!";
	String link = "m2014-07-03_165817_B-Richter";
	
	asyncTask.execute(tID, title, content, link);	
	...
\end{lstlisting}

When the new message has been created the AsyncSendMessage will update the ListView in the ActivityMessages through the MessageArrayAdapter and show the new message on top of the list.

\subsubsection{AsyncSetMessage}
The AsyncSetMessage will update an existing message at the beapDB as well as at the SQLite database. It works pretty much like the AsyncSendMessage, except that the only argument is the modified BlubbMessage object. 

\subsection{AsyncTasks of the ActivityLogin}\label{subsec:AsyncLogin}
\subsubsection{AsyncLogin}
An AsyncLogin will perform a login at the beap server and store the SessionInfo in the SessionManager, so all requests at the beapDB can use the session ID to verify the identity of the user. To execute a login a String with the username and one with the password is needed as arguments in the \code{execute()} method.
After a successful login the ActivityThreads will be started and if the user has checked the CheckBox the username and password will be stored in the shared preferences. If a PasswordInitException has been caught, the password init dialog will be started by calling \code{showPasswordInitDialog()} of the current ActivityLogin. 
\subsubsection{AsyncResetPassword}
With a AsyncResetPassword object the password of a user can be reset. It needs four Strings as arguments in the \code{execute()} method, the username, old password, new password and the new password again to confirm it. After it has finished its task, it will start the ActivityLogin with the LoginType LOGIN. So if the user has entered a wrong new password this will be recognized immediately. Has a init password dialog started the task, it will be closed.

\section{Manifest}
Every application must have an \code{AndroidManifest.xml} file (with precisely that name) in its root directory. The manifest file presents essential information about an app to the Android system \citep{aDefManifest}.
In \appname  the manifest declares the ActivityThread as the main activity, the activities ActivityLogin, ActivityMessages and ActivitySettings. The PullService is also defined in the manifest. Since the app uses a custom Application class this is stated within the application tag.
The permissions needed for the app are the 'internet' and 'vibrate' permissions. Of course is a network access mandatory to communicate with others and the vibration of the android device is used to inform the user about new messages.

\chapter{User documentation}

\section{Thread overview}

\subsection{Show Threads}
blubb
\subsection{Thread details}

\subsection{Create a new thread}

\subsection{Modify thread title and description}

\section{Thread messages}

\subsection{Show messages for a thread}

\subsection{Write a new message}

\subsection{Modify a message}

\subsection{Reply to a message}

\subsection{Read replies}

\section{Settings and login}

\subsection{login}

\subsection{logout}

\subsection{Reset the password}

\subsection{First login}

\subsection{Settings}

\subsection{Stay logged in}

\chapter{Valuation}

\chapter{Prospect blubb}

\printbibliography
%\bibliography{thesisBib}

\appendix
\chapter{Requests and responses from beap}
\lstset{language=JavaScript}
\section{Session requests}
To try an url please change parameter to valid values, e.g. the username and the password or the session id.
\subsubsection{Login}
BeapId=BeapSession\\
Action=login\\
appVers=1.5.0 rc1\\
uName="username"\\
uPwd="password"\\
\url{http://blubb.traeumtgerade.de:9980/?BeapId=BeapSession&Action=login&appVers=1.5.0 rc1&uName=Der-Praktikant&uPwd=xxxx}
\lstset{language=JavaScript}
\begin{lstlisting}
{ /* ReturnOkObj */
    "BeapStatus" : 200,
    "StatusDescr" : "OK",
    "sessInfo" : { 
        "sessId" : "5a379bd91cc16ca1dc01198f54fbb55d",
        "sessUser" : "Der-Praktikant",
        "sessRole" : "user",
        "sessActive" : true,
        "expires" : "2014-07-31T22:28:54.000+0200"
    }
}
\end{lstlisting}
\clearpage
\subsubsection{Check session}
BeapId=BeapSession\\
Action=check\\
sessId="session id"
\url{http://blubb.traeumtgerade.de:9980/?BeapId=BeapSession&Action=check&sessId=5a379bd91cc16ca1dc01198f54fbb55d}
\lstset{language=JavaScript}
\begin{lstlisting}
{ /* ReturnOkObj */
    "BeapStatus" : 200,
    "StatusDescr" : "OK",
    "Result" : 6,
    "sessInfo" : { 
        "sessId" : "5a379bd91cc16ca1dc01198f54fbb55d",
        "sessUser" : "Der-Praktikant",
        "sessRole" : "user",
        "sessActive" : true,
        "expires" : "2014-07-31T22:32:21.000+0200"
    }
}
\end{lstlisting}

\subsubsection{Session refresh}

BeapId=BeapSession\\
Action=refresh\\
sessId="session id"
\url{http://blubb.traeumtgerade.de:9980/?BeapId=BeapSession&Action=refresh&sessId=5a379bd91cc16ca1dc01198f54fbb55d}
\lstset{language=JavaScript}
\begin{lstlisting}
{ /* ReturnOkObj */
    "BeapStatus" : 200,
    "StatusDescr" : "OK",
    "Result" : 5,
    "sessInfo" : { 
        "sessId" : "5a379bd91cc16ca1dc01198f54fbb55d",
        "sessUser" : "Der-Praktikant",
        "sessRole" : "user",
        "sessActive" : true,
        "expires" : "2014-07-31T22:32:21.000+0200"
    }
}
\end{lstlisting}

\subsubsection{Logout}

BeapId=BeapSession\\
Action=logout\\
sessId="session id"
\url{http://blubb.traeumtgerade.de:9980/?BeapId=BeapSession&Action=logout&sessId=5a379bd91cc16ca1dc01198f54fbb55d}
\lstset{language=JavaScript}
\begin{lstlisting}
{ /* ReturnOkObj */
    "BeapStatus" : 200,
    "StatusDescr" : "OK",
    "sessInfo" : { 
        "sessId" : "",
        "sessUser" : "",
        "sessRole" : "",
        "sessActive" : false,
        "expires" : { 

        }
    }
}
\end{lstlisting}

\subsubsection{Reset Password}

BeapId=BeapSession\\
Action=setOwnPwd\\
uName="username"\\
uPwd="password"\\
newPwd1="new password"\\
newPwd2="new password again"\\
\url{http://blubb.traeumtgerade.de:9980/?BeapId=BeapSession&Action=setOwnPwd&uName="Der-Praktikant"&uPwd="xxxx"&newPwd1="xxxx"&newPwd2="xxxx"}
\lstset{language=JavaScript}
\begin{lstlisting}
{ /* ReturnOkObj */
    "BeapStatus" : 200,
    "StatusDescr" : "OK"
}
\end{lstlisting}

\section{Database requests}
\subsubsection{General request}
The queries to the beapDB have all the same request structure but different query strings and result objects. The following sections will focus on these queries and their results.
\\
BeapId=BeapDB\\
sessId="session id"\\
Action=query\\
queryStr="the query string"
\url{http://blubb.traeumtgerade.de:9980/?BeapId=BeapDB&sessId=5a379bd91cc16ca1dc01198f54fbb55d&Action=query&queryStr=tree.functions.getAllThreads(self)}

\begin{lstlisting}
{ /* ReturnOkObj */
    "BeapStatus" : 200,
    "StatusDescr" : "OK",
    "Result" : [  
        { 
            "tId" : "t2014-06-18_115628_S-Gross",
            "tType" : "Thread",
            "tCreator" : "S-Gross",
            "tCreatorRole" : "admin",
            "tPath" : [  
                "Thread"
            ],
            "tDate" : "2014-06-18T11:56:28.313Z",
            "tTitle" : "Off-Topic-Ge[Blubb]er",
            "tDescr" : "einfach nur zum blubbern.",
            "tStatus" : "open",
            "tMsgCount" : 1
        },
        ...
    ],
    "sessInfo" : { 
        "sessId" : "5a379bd91cc16ca1dc01198f54fbb55d",
        "sessUser" : "Der-Praktikant",
        "sessRole" : "user",
        "sessActive" : true,
        "expires" : "2014-07-31T23:00:25.000+0200"
    }
}
\end{lstlisting}

\subsubsection{Get all threads}
Query=tree.functions.getAllThreads(self)\\
The result is an array of threads.
\begin{lstlisting}
    "Result" : [  
        { 
            "tId" : "t2014-06-18_115628_S-Gross",
            "tType" : "Thread",
            "tCreator" : "S-Gross",
            "tCreatorRole" : "admin",
            "tPath" : [  
                "Thread"
            ],
            "tDate" : "2014-06-18T11:56:28.313Z",
            "tTitle" : "Off-Topic-Ge[Blubb]er",
            "tDescr" : "einfach nur zum blubbern.",
            "tStatus" : "open",
            "tMsgCount" : 1
        },
        { 
            "tId" : "t2014-06-20_145439_S-Gross",
            "tType" : "Thread",
            "tCreator" : "S-Gross",
            "tCreatorRole" : "admin",
            "tPath" : [  
                "Thread"
            ],
            "tStatus" : "solved",
            "tDate" : "2014-06-20T14:54:39.775Z",
            "tTitle" : "Bug: Login-Screen",
            "tDescr" : "...",
            "tMsgCount" : 2
        },
        ...
        
    ]
\end{lstlisting}
\clearpage
\subsubsection{Create a thread}
tree.functions.createThread(self,"tTitle","tDescription")\\
The result is the created thread.
\begin{lstlisting}
    "Result" : 
    	{ 
            "tId" : "t2014-06-18_115628_S-Gross",
            "tType" : "Thread",
            "tCreator" : "S-Gross",
            "tCreatorRole" : "admin",
            "tPath" : [  
                "Thread"
            ],
            "tDate" : "2014-06-18T11:56:28.313Z",
            "tTitle" : "Off-Topic-Ge[Blubb]er",
            "tDescr" : "einfach nur zum blubbern.",
            "tStatus" : "open",
            "tMsgCount" : 1
        }
\end{lstlisting}

\subsubsection{Modify a thread}
Query=tree.functions.setThread( self,"tId","tTitle","tDescription","tStatus")
\\The result is the modified thread.
\begin{lstlisting}
    "Result" : 
    	{ 
            "tId" : "t2014-06-18_115628_S-Gross",
            "tType" : "Thread",
            "tCreator" : "S-Gross",
            "tCreatorRole" : "admin",
            "tPath" : [  
                "Thread"
            ],
            "tDate" : "2014-06-18T11:56:28.313Z",
            "tTitle" : "Off-Topic-Ge[Blubb]er",
            "tDescr" : "einfach nur zum blubbern.",
            "tStatus" : "open",
            "tMsgCount" : 1
        }
\end{lstlisting}

\subsubsection{Quick check}
Query=tree.functions.quickCheck(self)\\
The result will be a simple array of integer. The first integer represents the number of threads at the database and the second the number of messages.

\begin{lstlisting}
   "Result" : [  
        17,
        102
    ]
\end{lstlisting}

\subsubsection{Get messages for a thread}
Query=tree.functions.getMsgsForThread(self,"tId")\\
The result will be an array of messages.
\begin{lstlisting}
   "Result" : [  
        { 
            "mId" : "m2014-06-18_122643_S-Gross",
            "mType" : "Message",
            "mCreator" : "S-Gross",
            "mCreatorRole" : "admin",
            "mDate" : "2014-06-18T12:26:43.422Z",
            "mThread" : [  
                "t2014-06-18_115628_S-Gross"
            ],
            "mTitle" : "  ",
            "mContent" : "Bitte Alle ToDo-Threads mit ..."
        },
        ...
    ]
\end{lstlisting}

\subsubsection{Create a message}
Query=tree.functions.createMsg(self,"tId","mTitle","mContent","mLink")\\
The result is the created message.
\begin{lstlisting}
   "Result" :   
        { 
            "mId" : "m2014-06-18_122643_S-Gross",
            "mType" : "Message",
            "mCreator" : "S-Gross",
            "mCreatorRole" : "admin",
            "mDate" : "2014-06-18T12:26:43.422Z",
            "mThread" : [  
                "t2014-06-18_115628_S-Gross"
            ],
            "mTitle" : "  ",
            "mContent" : "Bitte Alle ToDo-Threads mit ..."
        }
\end{lstlisting}

\subsubsection{Modify a message}
Query=tree.functions.setMsg(self,"mId","mTitle","mContent","mLink")\\
The result is the modified message.
\begin{lstlisting}
   "Result" :   
        { 
            "mId" : "m2014-06-18_122643_S-Gross",
            "mType" : "Message",
            "mCreator" : "S-Gross",
            "mCreatorRole" : "admin",
            "mDate" : "2014-06-18T12:26:43.422Z",
            "mThread" : [  
                "t2014-06-18_115628_S-Gross"
            ],
            "mTitle" : "  ",
            "mContent" : "Bitte Alle ToDo-Threads mit ..."
        }
\end{lstlisting}



\end{document}
